% !TEX root = /media/ueslei/Ueslei/INPE/PCI/Guia_COAWST/main.tex
\chapterimage{header.jpg}
\chapter*{Agradecimentos}

\addcontentsline{toc}{chapter}{Agradecimentos}
\bigskip


\noindent Ao CNPq pela Bolsa de Capacitação Institucional do Instituto Nacional de Pesquisas Espaciais, processo 301110/2017-4 
          fornecida a U. A. Sutil e também pela bolsa do programa de Produtividade em Pesquisa concedida a
          L. P. Pezzi (CNPq 304009/2016-4) e ao Programa Antártico Brasileiro (CNPq 443013/2018-7) 
\bigskip

\noindent À CAPES pelo fomento no projeto Advanced Studies in Medium and High Latitudes Oceanography (23038.004304/2014-28). 
\bigskip

\noindent Ao Trond Kristiansen (\textcolor{bleu_cite}{\textit{\href{https://github.com/trondkr/model2roms}{https://github.com/trondkr/model2roms}}}) por disponibilizar o 
          pacote \textit{model2roms}, que auxiliou no desenvolvimento para gerar as condições do ROMS. 
\bigskip

\noindent À Kate Hedström pela ajuda e suporte para aprender sobre o modelo de gelo marinho.
\bigskip

\noindent Ao Matheus Fagundes pela ajuda no Python. 
\bigskip

\noindent Ao João Hackerott pela ajuda em compilar o WPS no cluster Kerana e nas inúmeras dúvidas com o WRF.
\bigskip

\noindent Ao Jonas Takeo Carvalho por contribuir com dicas sobre o SWAN. 
\bigskip

\noindent À toda equipe de modelagem que desenvolveu o ROMS e distribuiu de forma aberta os códigos fontes, 
          especialmente ao Hernan G. Arango.
\bigskip

\noindent Ao John Warner e todos os idealizadores e colaboradores do COAWST por desenvolverem e disponibilizarem os códigos de forma livre e gratuita.
\bigskip

\noindent Ao Mathias Legrand, Vel e Andrea Hidalgo pelo template no \LaTeX.
\bigskip