% !TEX root = /media/ueslei/Ueslei/INPE/PCI/Projetos/Guia_COAWST/main.tex
\chapterimage{header.jpg}
\chapter*{Nota dos autores}
\bigskip
\noindent Este guia foi desenvolvido para auxiliar usuários novos com a familiarização e utilização do sistema 
          de modelagem regional acoplada (COAWST). A principal ideia é ensinar ao leitor as etapas necessárias 
          para utilizar o COAWST, desde a sua instalação, simulação de um caso teste e a configuração de um 
          projeto. Para atingir este objetivo, utilizamos diversas linguagens de programação, como Fortran, 
          Python e MATLAB. Futuramente pretendemos adaptar todas as rotinas (scripts) para linguagem computacionais 
          que sejam de uso livres e gratuitas.
\bigskip

\noindent Quando começamos a escrever este guia, queríamos passar para o leitor a nossa experiência em utilizar o 
          sistema de modelagem numérica considerado o estado da arte na área, de modo que fosse possível, ao longo 
          da leitura, entender o funcionamento e a aplicação dele, unindo a teoria com a prática.
\bigskip

\noindent Porém, uma grande dificuldade neste processo foi a geração das condições de contorno e inicial do modelo 
          regional oceânico, o Regional Ocean Modeling System (ROMS), que depende de softwares pagos. Para contornar este 
          problema escolhemos trabalhar com o pacote \textit{model2roms}. Esse conjunto de rotinas foi desenvolvido 
          em linguagem Python e Fortran por Trond Kristiansen 
          (\textcolor{bleu_cite}{\href{http://www.trondkristiansen.com}{http://www.trondkristiansen.com}}).
\bigskip

\noindent Destacamos, também, que em alguns capítulos o leitor encontrará como utilizar o COAWST em um cluster com
          a arquitetura que está disponível para o uso do Laboratório de Estudos do Oceano e da Atmosfera (LOA) da Coordenação
          Geral de Observação da Terra (OBT) do Instituto Nacional de Pesquisas Espaciais (INPE). Este é um sistema de 
          computação de alta performance que permite o paralelismo de operações numéricas. Neste caso, o guia poderá servir 
          somente como inspiração e valerá o conhecimento prévio do leitor em aplicar o COAWST no seu próprio sistema computacional.
\bigskip

\noindent Nesta segunda edição, destacamos a atualização para o COAWST v.3.4, a introdução de um capítulo sobre
          o modelo de gelo marinho, e a renovação do pacote de ferramentas \textit{model2roms} para 
          gerar os dados de entrada do modelo e a reorganização estrutural Guia.
\bigskip
\bigskip

\noindent Para ler a primeira edição (\cite{Sutil2018}), acesse: \textcolor{bleu_cite}{\href{http://www.doi.org/10.13140/RG.2.2.31726.87363}{http://www.doi.org/10.13140/RG.2.2.31726.87363}}.
\bigskip

\noindent Para citar a primeira edição, use a seguinte referência:
\bigskip

\noindent SUTIL, U. A.; PEZZI, L. P. Guia prático para utilização do COAWST. São José dos Campos: INPE, 2018. 86 p. IBI: <8JMKD3MGP3W34R/3RQSQ2L>. ISBN: <978-85-17-00093-5>. Disponível em: <\textcolor{bleu_cite}{\href{http://urlib.net/rep/8JMKD3MGP3W34R/3RQSQ2L}{http://urlib.net/rep/8JMKD3MGP3W34R/3RQSQ2L}}>. 
\bigskip

\noindent Para citar a segunda edição, use a seguinte referência:
\bigskip

\noindent SUTIL, U. A.; PEZZI, L. P. Guia prático para utilização do COAWST - 2ª Edição. São José dos Campos: INPE, 2018. 100 p. IBI: <8JMKD3MGP3W34R/3RQSQ2L>. ISBN: <978-85-17-00093-5>. Disponível em: <\textcolor{bleu_cite}{\href{http://urlib.net/rep/8JMKD3MGP3W34R/3RQSQ2L}{http://urlib.net/rep/8JMKD3MGP3W34R/3RQSQ2L}}>. 

\bigskip
\bigskip

\noindent Desejamos uma boa leitura e sucesso na sua pesquisa.
\begin{flushright}
\noindent Os autores.
\end{flushright}
\bigskip
