% !TEX root = /media/ueslei/Ueslei/INPE/PCI/Projetos/Guia_COAWST/main.tex
\chapterimage{header.jpg}
\chapter*{Author's note}
\bigskip
\noindent This guide is designed to assist new users to use the Coupled Ocean-Atmosphere-Wave-Sediment Transport System (COAWST). The main idea behind this 
guide is to teach the necessary steps to use COAWST, beginning with its installation, then a simulation of a test case and the configuration of a
project. To achieve this goal, we use several programming languages, such as Fortran, Python and MATLAB. In the future we intend to adapt all scripts
to a free programming language.
\bigskip

\noindent When we started writing this guide, we wanted to pass on our experience of using a numerical modeling system that is considered the state of the art in our field, 
throughout reading, understanding how it works and how to use the COAWST, allying theory with practice.
\bigskip

\noindent However, a major difficulty in this process was the generation of the boundary and initial conditions of the oceanic model, the Regional Ocean Modeling System (ROMS),
which rely on paid software. To get around this problem we chosed to work with the \textit {model2roms} toolbox package. This set of routines was developed in Python and 
Fortran language by Trond Kristiansen (\textcolor{bleu_cite} {\href{http://www.trondkristiansen.com}{http://www.trondkristiansen.com}}).
\bigskip

\noindent We emphasize that in some chapters, the reader will find how to use the COAWST in a cluster that is available for use by the Ocean and Atmosphere 
Studies Laboratory (LOA) of the National Institute for Space Research (INPE). This is a system of high performance computing that allows parallel numerical operations. 
In this case, the guide may serve only as inspiration and the reader's prior knowledge in applying COAWST in his own cluster system will be worthwhile.
\bigskip

\noindent In this third edition, we highlight the update to COAWST v.3.6, a new chapter about how to compile the model in a computer without parallel architecture.
\bigskip

\noindent To cite the \textbf{third edition}, use the following:
\bigskip

\bigskip
\pagebreak 

\noindent To cite the \textbf{first edition}, use the following:
\bigskip

\noindent SUTIL, U. A.; PEZZI, L. P. Guia prático para utilização do COAWST. São José dos Campos: INPE, 2018. 86 p. IBI: <8JMKD3MGP3W34R/3RQSQ2L>. ISBN: <978-85-17-00093-5>. Avaliable at: <\textcolor{bleu_cite}{\href{http://urlib.net/rep/8JMKD3MGP3W34R/3RQSQ2L}{http://urlib.net/rep/8JMKD3MGP3W34R/3RQSQ2L}}>. 
\bigskip

\noindent To cite the \textbf{second edition}, use the following:
\bigskip

\noindent SUTIL, U. A.; PEZZI, L. P. Guia prático para utilização do COAWST - 2ª Edição. São José dos Campos: INPE, 2019. 100 p. IBI: <8JMKD3MGPCW/3DT298SL>. ISBN: <978-85-17-00098-0>. Avaliable at: <\textcolor{bleu_cite}{\href{http://urlib.net/rep/8JMKD3MGP3W34R/3TUTUJB}{http://urlib.net/rep/8JMKD3MGP3W34R/3TUTUJB}}>. 

\bigskip
\bigskip

\noindent We wish you a good reading and success in your research.
\begin{flushright}
\noindent The authors.
\end{flushright}
\bigskip
