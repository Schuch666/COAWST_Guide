% !TEX root = /media/ueslei/Ueslei/INPE/PCI/Projetos/Guia_COAWST/main.tex
\chapterimage{header.jpg}
\chapter{\large{Papers of the Ocean and Atmosphere Studies Laboratory (LOA)}}
\bigskip

%%%%%%%%%%%%% ARTIGO  %%%%%%%%%%%%%%%%

\noindent \begin{center}
\textbf{Ocean-Atmosphere Interactions in an Extratropical Cyclone in the Southwest Atlantic}
\bigskip

\noindent U. A. Sutil, L. P. Pezzi, R. C. M. Alves and A. B. Nunes
\bigskip

\noindent \textbf{Abstract}\end{center}
\bigskip

\noindent This work shows an investigation of the behavior of heat fluxes in the processes of ocean-atmosphere interaction
during the passage of an Extra-tropical Cyclone (EC) in the Southwest Atlantic in September 2006 using a coupled
regional model’s system. A brief evaluation of the simulated data is done by comparison with air and sea surface temperature (SST) data, wind speed, sea level pressure. This comparison showed that both model simulations present some
differences (mainly, the wind), nevertheless the simulated variables show quite satisfactory results, therefore allowing a
good analysis of the ocean-atmosphere interaction processes. The simulated thermal gradient increases the ocean’s heat
fluxes into the atmosphere in the cold sector of the cyclone and through the convergence of low level winds the humidity
is transported to higher levels producing precipitation. The coupled system showed a greater ability to simulate the intensity and trajectory of the cyclone, compared to the simulation of the atmospheric model.
\bigskip

\noindent \textcolor{black}{\fullcite{Sutil2019}}
\bigskip

\noindent Avaliable at: \textcolor{bleu_cite}{\href{http://www.anuario.igeo.ufrj.br/2019\_01/2019\_1\_525\_535.pdf}{\textit{http://www.anuario.igeo.ufrj.br/2019\_01/2019\_1\_525\_535.pdf}}}
\bigskip

%%%%%%%%%%%%% ARTIGO  %%%%%%%%%%%%%%%%
\newpage
\noindent \begin{center}
\textbf{Low connectivity compromises the conservation of reef fishes by marine protected areas in the tropical South Atlantic}
\bigskip

\noindent C. A. K. Endo, D. F. M. Gherardi, L. P. Pezzi and L. N. Lima
\bigskip

\noindent \textbf{Abstract}\end{center}
\bigskip

\noindent The total spatial coverage of Marine Protected Areas (MPAs) within the Brazilian Economic Exclusive Zone (EEZ) has recently achieved the quantitative requirement of the Aichii 
          Biodiversity Target 11. However, the distribution of MPAs in the Brazilian EEZ is still unbalanced regarding the proportion of protected ecosystems, protection goals and management types. 
          Moreover, the demographic connectivity between these MPAs and their effectiveness regarding the maintenance of biodiversity are still not comprehensively understood. An individual-based
          modeling scheme coupled with a regional hydrodynamic model of the ocean is used to determine the demographic connectivity of reef fishes based on the widespread genus Sparisoma found in 
          the oceanic islands and on the Brazilian continental shelf between 10° N and 23° S. Model results indicate that MPAs are highly isolated due to extremely low demographic connectivity. 
          Consequently, low connectivity and the long distances separating MPAs contribute to their isolation. Therefore, the current MPA design falls short of its goal of maintaining the demographic
          connectivity of Sparisoma populations living within these areas. In an extreme scenario in which the MPAs rely solely on protected populations for recruits, it is unlikely that they will be 
          able to effectively contribute to the resilience of these populations or other reef fish species sharing the same dispersal abilities. Results also show that recruitment occurs elsewhere 
          along the continental shelf indicating that the protection of areas larger than the current MPAs would enhance the network, maintain connectivity and contribute to the conservation of reef fishes.
\bigskip

\noindent \textcolor{black}{\fullcite{Endo2019}}
\bigskip

\noindent Avaliable at: \textcolor{bleu_cite}{\href{https://www.nature.com/articles/s41598-019-45042-0}{\textit{https://www.nature.com/articles/s41598-019-45042-0}}}
\bigskip


%%%%%%%%%%%%% ARTIGO  %%%%%%%%%%%%%%%%
\newpage
\noindent \begin{center}
\textbf{An Investigation of Ocean Model Uncertainties Through Ensemble Forecast Experiments in the Southwest Atlantic Ocean}
\bigskip

\noindent L. N. Lima, L. P. Pezzi, S. G. Penny and C. A. S. Tanajura
\bigskip

\noindent \textbf{Abstract}\end{center}
\bigskip

\noindent Ocean general circulation models even with realistic behavior still incorporate large uncertainties from external forcing. This study involves the realization of ensemble experiments using a 
          regional model configured for the Southwest Atlantic Ocean to investigate uncertainties derived from the external forcing such as the atmosphere and bathymetry. The investigation is based on
          perturbing atmospheric surface fluxes and bathymetry through a series of ensemble experiments. The results showed a strong influence of the South Atlantic Convergence Zone on the underlying ocean, 
          7 days after initialization. In this ocean region, precipitation and radiation flux perturbations notably impacted the sea surface salinity and sea surface temperature, by producing values of 
          ensemble spread that exceeded 0.08 and 0.2 °C, respectively. Wind perturbations extended the impact on currents at surface, with the spread exceeding 0.1 m/s. The ocean responded faster to 
          the bathymetric perturbations especially in shallow waters, where the dynamics are largely dominated by barotropic processes. Ensemble spread was the largest within the thermocline layer and in
          ocean frontal regions after a few months, but by this time, the impact on the modeled ocean obtained from either atmospheric or bathymetric perturbations was quite similar, with the internal 
          dynamics dominating over time. In the vertical, the sea surface temperature exhibited high correlation with the subsurface temperature of the shallowest model levels within the mixed layer.
          Horizontal error correlations exhibited strong flow dependence at specific points on the Brazil and Malvinas Currents. This analysis will be the basis for future experiments using ensemble-based 
          data assimilation in the Southwest Atlantic Ocean.
\bigskip

\noindent \textcolor{black}{\fullcite{Lima2019}}
\bigskip

\noindent Avaliable at: \textcolor{bleu_cite}{\href{https://agupubs.onlinelibrary.wiley.com/doi/epdf/10.1029/2018JC013919}{\textit{hhttps://agupubs.onlinelibrary.wiley.com/doi/epdf/10.1029/2018JC013919}}}
\bigskip

%%%%%%%%%%%%% ARTIGO  %%%%%%%%%%%%%%%%
\newpage
\noindent \begin{center}
\textbf{Coupled ocean-atmosphere forecasting at short and medium time scales}
\bigskip

\noindent J. Pullen, R. Allard, H. Seo, A. J. Miller, S. Chen, L. P. Pezzi, T. Smith, P. Chu, J. Alves and R. Caldeira
\bigskip

\noindent \textbf{Abstract}\end{center}
\bigskip

\noindent Recent technological advances over the past few decades have enabled the development of fully coupled atmosphere-ocean modeling prediction systems which are used today to support 
          short-term (days to weeks) and medium-term (10-21 days) needs for both the operational and research communities. Utilizing several coupled modeling systems we overview the coupling 
          framework, including model components and grid resolution considerations, as well as the coupling physics by examining heat fluxes between atmosphere and ocean, momentum transfer, 
          and freshwater fluxes. These modeling systems can be run as fully coupled atmosphere-ocean and atmosphere-ocean-wave configurations. Examples of several modeling systems applied to 
          complex coastal regions including Madeira Island, Adriatic Sea, Coastal California, Gulf of Mexico, Brazil, and the Maritime Continent are presented. In many of these studies, a 
          variety of field campaigns have contributed to a better understanding of the underlying physics associated with the atmosphere-ocean feedbacks. Examples of improvements in predictive 
          skill when run in coupled mode versus standalone are shown. Coupled model challenges such as model initialization, data assimilation, and earth system prediction are discussed.
\bigskip

\noindent \textcolor{black}{\fullcite{Pullen2017}}
\bigskip

\noindent Avaliable at: \textcolor{bleu_cite}{\href{http://meteora.ucsd.edu/$\sim$miller/papers/TheSea\_Chapter23.html}{\textit{http://meteora.ucsd.edu/$\sim$miller/papers/TheSea\_Chapter23.html}}}
\bigskip

%%%%%%%%%%%%%%%% ARTIGO  %%%%%%%%%%%%%%
\newpage
\bigskip

\noindent \begin{center} \textbf{Regional modeling of the water masses and circulation annual variability at the Southern Brazilian Continental Shelf}
\bigskip

\noindent L. F. Mendonça, R. B. Souza, C. R. C. Aseff, L. P. Pezzi, O. O. Möller and R. C. M. Alves
\bigskip

\noindent \textbf{Abstract}\end{center}
\bigskip

\noindent The Southern Brazilian Continental Shelf (SBCS) is one of the more productive areas for fisheries in Brazilian waters. The water masses and the dynamical processes of the region 
          present a very seasonal behavior that imprint strong effects in the ecosystem and the weather of the area and its vicinity. This paper makes use of the Regional Ocean Modeling System (ROMS) 
          for studying the water mass distribution and circulation variability in the SBCS during the year of 2012. Model outputs were compared to in situ, historical observations and to satellite data. 
          The model was able to reproduce the main thermohaline characteristics of the waters dominating the SBCS and the adjacent region. The mixing between the Subantarctic Shelf Water and the Subtropical
          Shelf Water, known as the Subtropical Shelf Front (STSF), presented a clear seasonal change in volume. As a consequence of the mixing and of the seasonal oscillation of the STSF position, 
          the stability of the water column inside the SBCS also changes seasonally. Current velocities and associated transports estimated for the Brazil Current (BC) and for the Brazilian Coastal Current 
          (BCC) agree with previous measurements and estimates, stressing the fact that the opposite flow of the BCC occurring during winter in the study region is about 2 orders of magnitude smaller than that
           of the BC. Seasonal maps of simulated Mean Kinetic Energy and Eddy Kinetic Energy demonstrate the known behavior of the BC and stressed the importance of the mean coastal flow off Argentina throughout the year.
\bigskip

\noindent \fullcite{Mendonca2017}
\bigskip

\noindent Avaliable at: \textcolor{bleu_cite}{\href{https://agupubs.onlinelibrary.wiley.com/doi/abs/10.1002/2016JC011780}{\textit{https://agupubs.onlinelibrary.wiley.com/doi/abs/10.1002/2016JC011780}}}

\bigskip


%%%%%%%%%%%%%%%% ARTIGO %%%%%%%%%%%%%%
\newpage
\bigskip

\noindent \begin{center} \textbf{The Influence of Sea Ice Dynamics on the Climate Sensitivity and Memory to Increased Antarctic Sea Ice}
\bigskip

\noindent C. K. Parise, L. P. Pezzi, K. I. Hodges and F. Justino
\bigskip

\noindent \textbf{Abstract}\end{center}
\bigskip

\noindent The study analyzes the sensitivity and memory of the Southern Hemisphere coupled climate system to increased Antarctic sea ice (ASI), taking into account the persistence of the sea ice maxima
          in the current climate. The mechanisms involved in restoring the climate balance under two sets of experiments, which differ in regard to their sea ice models, are discussed. The experiments 
          are perturbed with extremes of ASI and integrated for 10 yr in a large 30-member ensemble. The results show that an ASI maximum is able to persist for ~4 yr in the current climate, 
          followed by a negative sea ice phase. The sea ice insulating effect during the positive phase reduces heat fluxes south of 60\degree S, while at the same time these are intensified at the 
          sea ice edge. The increased air stability over the sea ice field strengthens the polar cell while the baroclinicity increases at midlatitudes. The mean sea level pressure is reduced 
          (increased) over high latitudes (midlatitudes), typical of the southern annular mode (SAM) positive phase. The Southern Ocean (SO) becomes colder and fresher as the sea ice melts mainly through 
          sea ice lateral melting, the consequence of which is an increase in the ocean stability by buoyancy and mixing changes. The climate sensitivity is triggered by the sea ice insulating process and 
          the resulting freshwater pulse (fast response), while the climate equilibrium is restored by the heat stored in the SO subsurface layers (long response). It is concluded that the time needed for 
          the ASI anomaly to be dissipated and/or melted is shortened by the sea ice dynamical processes.

\bigskip
\noindent \fullcite{Parise2015}
\bigskip

\noindent Avaliable at: \textcolor{bleu_cite}{\href{https://journals.ametsoc.org/doi/10.1175/JCLI-D-14-00748.1}{\textit{https://journals.ametsoc.org/doi/10.1175/JCLI-D-14-00748.1}}}

\bigskip

%%%%%%%%%%%%%%%% ARTIGO %%%%%%%%%%%%%%
\newpage
\bigskip

\noindent \begin{center} \textbf{Modeling the spawning strategies and larval survival of the Brazilian sardine (\textit{Sardinella brasiliensis})}
\bigskip

\noindent D. F. Dias, L. P. Pezzi, D. F. M. Gherardi and R. Camargo
\bigskip

\noindent \textbf{Abstract}\end{center}
\bigskip

\noindent An Individual Based Model (IBM), coupled with a hydrodynamic model (ROMS), was used to investigate the spawning strategies and larval survival of the Brazilian Sardine in the South Brazil Bight (SBB). 
          ROMS solutions were compared with satellite and field data to assess their representation of the physical environment. Two spawning experiments were performed for the summer along six years, 
          coincident with ichthyoplankton survey cruises. In the first one, eggs were released in spawning habitats inferred from a spatial model. The second experiment simulated a random spawning to test
          the null hypothesis that there are no preferred spawning sites. Releasing eggs in the predefined spawning habitats increases larval survival, suggesting that the central-southern part of the SBB is
          more suitable for larvae development because of its thermodynamic characteristics. The Brazilian sardine is also capable of exploring suitable areas for spawning, according to the interannual
          variability of the SBB. The influence of water temperature, the presence of Cape Frio upwelling, and surface circulation on the spawning process was tested. The Cape Frio upwelling plays an important 
          role in the modulation of Brazilian sardine spawning zones over SBB because of its lower than average water temperature. This has a direct influence on larval survival and on the interannual variability
          of the Brazilian sardine spawning process. The hydrodynamic condition is crucial in determining the central-southern part of SBB as the most suitable place for spawning because it enhances simulated 
          coastal retention of larvae.
\bigskip

\noindent \fullcite{Dias2014}
\bigskip

\noindent Avaliable at: \textcolor{bleu_cite}{\href{http://www.iag.usp.br/pos/meteorologia/biblio/modeling-spawning-strategies-and-larval-survival-brazilian-sardine-sardinella-br}{\textit{http://www.iag.usp.br/pos/meteorologia/biblio/modeling-spawning-strategies-and-larval-survival-brazilian-sardine-sardinella-br}}}


\bigskip

%%%%%%%%%%%%%%%% ARTIGO %%%%%%%%%%%%%%
\newpage
\bigskip

\noindent \begin{center} \textbf{Sea surface temperature anomalies driven by oceanic local forcing in the Brazil-Malvinas Confluence}
\bigskip

\noindent I. P. da Silveira and L. P. Pezzi
\bigskip

\noindent \textbf{Abstract}\end{center}
\bigskip

\noindent Sea surface temperature (SST) anomaly events in the Brazil-Malvinas Confluence (BMC) were investigated through wavelet analysis and numerical modeling. Wavelet analysis was applied to recognize 
          the main spectral signals of SST anomaly events in the BMC and in the Drake Passage as a first attempt to link middle and high latitudes. The numerical modeling approach was used to clarify the 
          local oceanic dynamics that drive these anomalies. Wavelet analysis pointed to the 8–12-year band as the most energetic band representing remote forcing between high to middle latitudes. Other 
          frequencies observed in the BMC wavelet analysis indicate that part of its variability could also be forced by low-latitude events, such as El Niño. Numerical experiments carried out for the 
          years of 1964 and 1992 (cold and warm El Niño-Southern Oscillation (ENSO) phases) revealed two distinct behaviors that produced negative and positive sea surface temperature anomalies on the 
          BMC region. The first behavior is caused by northward cold flow, Río de la Plata runoff, and upwelling processes. The second behavior is driven by a southward excursion of the Brazil Current 
          (BC) front, alterations in Río de la Plata discharge rates, and most likely by air-sea interactions. Both episodes are characterized by uncoupled behavior between the surface and deeper layers.
\bigskip

\noindent \fullcite{Silveira2014}
\bigskip

\noindent Avaliable at: \textcolor{bleu_cite}{\href{https://link.springer.com/article/10.1007\%2Fs10236-014-0699-4}{\textit{https://link.springer.com/article/10.1007\%2Fs10236-014-0699-4}}}

\bigskip
